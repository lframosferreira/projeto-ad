\documentclass{article}

\usepackage[english]{babel}

\usepackage[letterpaper,top=2cm,bottom=2cm,left=3cm,right=3cm,marginparwidth=1.75cm]{geometry}

\usepackage{amsmath}
\usepackage{graphicx}
\usepackage[colorlinks=true, allcolors=blue]{hyperref}
\usepackage{caption}
\usepackage{float}
\usepackage{csquotes}

\title{Projeto Final de Aprendizado Descritivo}
\author{Luís Felipe Ramos Ferreira \\  Igor Lacerda iFaria da Silva \\ Matheus
    Tiago Pimenta de Souza}
\date{}

\begin{document}
\maketitle

\section{Introdução}

O uso de ciência de dados e estatística para analisar esportes é algo que vem
crescendo cada vez mais nos últimos anos. Em
particular, o futebol têm sido um desses
esportes~\cite{takvorian2021beautiful}. A própria UFMG ofertou no
ano passado e novamente neste semestre a disciplina `Ciência de Dados aplicada
ao futebol', o que mostra a relevância do tema. Diversas empresas que atuam na
área surgem a cada dia, e os times de futebol, no Brasil e no resto do
mundo, estão investimento em seus departamentos de dados e estatística.

Nesse sentido, nosso grupo optou por estudar e compreender melhor como funciona
o uso de análises estatísticas no futebol, dado o interesse geral pelo esporte,
e, para isso, nos propusemos a aplicar algoritmos de mineração de dados em
dados futebolísticos, sendo eles dados de súmula, dados de eventos ou até mesmo
dados de
\textit{tracking} dos jogadores, para compreender como as informações acerca do
jogo estão contidas dentro dos dados coletados e como isso pode ser utilizado a
favor das equipes.

Os dados de eventos, especialmente, costumam ser mais fáceis de lidar e mais
fáceis de acessar do que dados de \textit{tracking}, enquanto trazem muito mais
informações do que dados de súmula. Existem, atualmente, algumas bases
gratuitas de dados de evento de partidas, disponibilizadas por diferentes
empresas como \textit{Wyscout} e \textit{StasBomb}. Como a ideia é ter um
panorama geral de diversas partidas, campeonatos e jogadores, iremos utilizar
as bases de dados disponibilizadas sobre as 5 grandes ligas de futebol europeu
das temporadas 17/18 da empresa \textit{Wyscout}.

\subsection{Base de dados}

As bases de dados utilizadas na ferramenta consistirá na base principal
disponibilizada pela empresa \textit{Wyscout},
consistindo em uma base de dados de evento das 5 grandes ligas europeias na
temporada 17/18.

Cada empresa fornecedora de dados possuem seu próprio formato de representação
dos dados de evento. De modo a facilitar a mesclagem entre as bases de dados
utilizadas, iremos converter os dados coletados para uma representação geral
proposta por pesquisadores denominada
\href{https://socceraction.readthedocs.io/en/latest/documentation/spadl/spadl.html}{SPADL}.
A SPADL é uma boa escolha por ser uma representação concisa e fácil de
utilizar. Ela é uma representação tabular de cada evento da partida, onde cada
linha possui 12 colunas. A tabela abaixo ilustra o esquema de representação de
um evento segundo o formato SPADL.

\begin{table}[H]
    \centering
    \begin{tabular}{|l|l|}
        \hline
        \textbf{Atributo} & \textbf{Descrição}
        \\
        \hline
        game\_id          & O ID do jogo no qual a ação foi realizada
        \\
        \hline
        period\_id        & O ID do período do jogo no qual a ação foi
        realizada
        \\
        \hline
        seconds           & O tempo de início da ação
        \\
        \hline
        player            & O jogador que realizou a ação
        \\
        \hline
        team              & O time do jogador
        \\
        \hline
        start\_x          & A localização x onde a ação começou
        \\
        \hline
        start\_y          & A localização y onde a ação começou
        \\
        \hline
        end\_x            & A localização x onde a ação terminou
        \\
        \hline
        end\_y            & A localização y onde a ação terminou
        \\
        \hline
        action\_type      & O tipo de ação (por exemplo, passe, chute, drible)
        \\
        \hline
        result            & O resultado da ação (por exemplo, sucesso ou falha)
        \\
        \hline
        bodypart          & A parte do corpo do jogador usada para a ação
        \\
        \hline
    \end{tabular}
    \caption{Descrição dos dados no formato SPADL}
\end{table}

\section{Implementação}

A linguagem escolhida para o desenvolvimento do trabalho foi
\href{https://www.python.org/}{\texttt{Python}} (versão 3.10.12), devida a seu
vasto ecossistema para ciência de dados e mineração de dados.

A manipulação dos dados foi feita com o uso de bibliotecas
de análise numérica como \href{https://numpy.org/}{\texttt{NumPy}} e
manipulação de \textit{dataframes} como
\href{https://pola.rs/}{\texttt{Polars}} e
\href{https://pandas.pydata.org/}{\texttt{Pandas}},
uma vez que se tratam de ferramentas extremamente completas que facilitaram o
desenvolvimento do projeto como um todo.

Para aplicar os algoritmos de descobertas de subgrupos, foi utilizado o pacote
\href{https://pysubgroup.readthedocs.io/en/latest/}{\texttt{pysubgroup}}, que
fornece uma aglomeração de algoritmos do estado da arte de descoberta de
subgrupos em um formato simples e leve para serem utilizados.

% falar aq de uqal pacote utilizamos para minerar as sequencias

Para organizar o ambiente de desenvolvimento, que englobava vários pacotes
diferentes, foi utilizado o gerenciador de pacotes
\href{https://www.anaconda.com/}{\texttt{Anaconda}}, o que facilitou o trabalho
com os pacotes de ciência de dados citados. O projeto final foi salvo em um
\href{https://github.com/lframosferreira/projeto-ad}{\texttt{repositório}}
no GitHub para fácil versionamento e organização de código. As instruções de
como
utilizar o que foi implementado estão descritas no arquivo \textit{README.md}
do repositório.

\section{Resultados}

gfdfd

\section{Conclusão}

fefre

\newpage
\bibliographystyle{plain}
\renewcommand{\refname}{Referências Bibliográficas}
\addcontentsline{toc}{section}{Referências Bibliográficas}
\bibliography{sample}

\end{document}