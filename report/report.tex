\documentclass{article}

\usepackage[english]{babel}

\usepackage[letterpaper,top=2cm,bottom=2cm,left=3cm,right=3cm,marginparwidth=1.75cm]{geometry}

\usepackage{amsmath}
\usepackage{graphicx}
\usepackage[colorlinks=true, allcolors=blue]{hyperref}
\usepackage{natbib}
\bibliographystyle{alpha}
\usepackage{caption}
\usepackage{float}
\usepackage{csquotes}

\title{Aprendizado Descritivo}
\author{Luís Felipe Ramos Ferreira \\  Igor Lacerda iFaria da Silva \\ Matheus Tiago Pimenta de Souza}

\begin{document}
\maketitle

\section{Introdução}

O projeto final da disciplina de Aprendizado de Descritivo teve como objetivo

\section{Implementação}

A linguagem escolhida para o desenvolvimento do trabalho foi
\href{https://www.python.org/}{\texttt{Python}} (versão 3.11.4), devida a sua
grande variedade de bibliotecas úteis para ciência de dados e aprendizado de
máquina.
A modelagem do algoritmo \textit{AdaBoost} foi feita com o uso de bibliotecas
de análise numérica como \href{https://numpy.org/}{\texttt{NumPy}} e manipulação de \textit{dataframes} como
\href{https://pola.rs/}{\texttt{Polars}},
uma vez que se tratam de ferramentas extremamente completas que facilitaram o
desenvolvimento do algoritmo.

Para organizar o ambiente de desenvolvimento, que englobava vários pacotes
diferentes, foi utilizado o gerenciador de pacotes
\href{https://www.anaconda.com/}{\texttt{Anaconda}}, o que facilitou o trabalho
com os pacotes de ciência de dados citados. O projeto final foi salvo em um
\href{https://github.com/lframosferreira/projeto-ad}{\texttt{repositório}}
no GitHub para fácil versionamento e organização de código. As instruções de
como
utilizar o que foi implementado estão descritas no arquivo \textit{README.md}
do repositório.

\subsection{Classificador}

\section{Conclusão}


\end{document}
